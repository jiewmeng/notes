\documentclass[12pt]{article}

\usepackage[T1]{fontenc}
\usepackage{setspace}
\usepackage[a4paper, margin=1.2cm, bottom=2.5cm]{geometry}
\usepackage{graphicx}
\usepackage{mdwlist}
\usepackage{amsmath, amssymb}
\usepackage{hyperref}
\usepackage{color, soul}

\graphicspath{{./_img/}}
\setlength{\parindent}{0cm}

\title{\textbf{Parallel Programming Models}}
\date{}

\begin{document}

\maketitle

\begin{spacing}{1.5}

\section{Parallel Programming Models}

\subsection{Classification}

\begin{itemize*}
	\item \textbf{Level of parallelism}: instruction, statement, procedural levels or parallel loops
	\item \textbf{Implicit/Explicit}
	\item \textbf{Processes/Threads}
	\item \textbf{Execution Mode}: SIMD, SPMD, Sync/Async
	\item \textbf{Communication modes}: Message passing/Shared variables
	\item \textbf{Syncronization mechanisms}
\end{itemize*}

\subsection{Program Parallelization}

\includegraphics[scale=0.7]{parallelization.png}

\includegraphics[scale=0.6]{parallelization-example.png}

\subsubsection{Decomposition}

\begin{itemize*}
	\item Sequential algorithm split into tasks/dependencies. Where task is a sequence of computations executed by single processor/core
	\item Static decomposition: program start/compile time
	\item Dynamic decomposition: during execution
	\item Goal
		\begin{itemize*}
			\item enough tasks to keep cores busy
			\item tasks $\ge$ cores
			\item Size of task $>>$ overhead of parallelism
		\end{itemize*}
\end{itemize*}

\subsubsection{Scheduling}

\begin{itemize*}
	\item Assignment of tasks to processes/threads (static/dynamic)
	\item Goal 
		\begin{itemize*}
			\item Load balancing (of computations/memory access (shared mem.)/communications (distributed mem.)) among processes/threads
			\item Efficient execution order
		\end{itemize*}
\end{itemize*}	


\subsubsection{Mapping}

\begin{itemize*}
	\item Assignment of processes/threads to execution units (cores/processors)
	\item Goal 
		\begin{itemize*}
			\item Balance utilization of execution units
			\item Minimize communications among processors
		\end{itemize*}
\end{itemize*}	

\section{Levels of Parallelism}

\includegraphics[scale=0.7]{levels-of-parallelism.png}

\subsection{Instruction parallelism}

Multiple instruction, but data dependencies inhibits parallel execution

\begin{itemize*}
	\item \textbf{Flow dependency ($W \rightarrow R$)}
	\item \textbf{Sequential consistency}: see lecture 3: sequential consistency
\end{itemize*}

\textbf{Data dependency graph}

\includegraphics[scale=0.7]{dep-graph.png}

\subsection{Data parallelism}

Same operation used for different data (SIMD)

MIMD/SPMD means one program executed by all processors in parallel using shared/distibuted memory

\subsection{Loop parallelism}

When no dependency between iterations of loop

\subsubsection{forall}

Think: each assignment in its only loop, executed before next assignment is run

\begin{verbatim}
forall (i = 1:4)
    a(i) = a(i) + 1
    b(i) = a(i-1) + b(i+1)
\end{verbatim}

becomes

\begin{verbatim}
for (i = 1:4)
    a(i) = a(i) + 1
for (i = 1:4)
    b(i) = a(i-1) + a(i+1)
\end{verbatim}

\subsubsection{dopar}

Iterations executed in parallel. Updates in 1 iteration transparent to others. 

\subsubsection{doall}


\subsection{Functional/Task Parallelism}

\begin{itemize*}
	\item Independent tasks can be executed in parallel
\end{itemize*}

\section{Parallel Programming Patterns}

\subsection{Fork-join}

\begin{itemize*}
	\item Create children threads, which work in parallel, with \texttt{fork}
	\item (Parent) Thread waits for children using \texttt{join} 
\end{itemize*}

\subsection{parbegin, parend}

\begin{itemize*}
	\item Similar to fork/join?
\end{itemize*}

\subsection{SPMD, SIMD}

\begin{itemize*}
	\item SIMD: single instruction executed sequentially by different threads on different data
	\item SPMD: Same program on different processors, but on different data. Threads work asynchronously with each other
\end{itemize*}

\subsection{Master-Slave/Worker}

\begin{itemize*}
	\item Master assigns work to workers
	\item Master responsible for coordination, initialization, output etc (bottleneck)
\end{itemize*}

\subsection{Client-Server}

\begin{itemize*}
	\item "Multiple program, multiple data"
	\item Server compute requests from multiple clients concurrently
\end{itemize*}

\subsection{Pipelining}

\begin{itemize*}
	\item Data partitioned into stream of data elements flows through pipeline threads
\end{itemize*}

\subsection{Task pools}

\begin{itemize*}
	\item Threads fixed (created statically)
\end{itemize*}

\subsection{Producer-Consumer}

\begin{itemize*}
	\item Producer produce data used as input to consumer
\end{itemize*}

\end{spacing}

\end{document}
