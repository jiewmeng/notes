\documentclass[12pt]{article}

\usepackage[T1]{fontenc}
\usepackage{setspace}
\usepackage[a4paper, margin=1.2cm, bottom=2.5cm]{geometry}
\usepackage{graphicx}
\usepackage{mdwlist}
\usepackage{amsmath}

\graphicspath{{./_img/}}
\setlength{\parindent}{0cm}

\title{\textbf{ Intro to Parallel Computing }}
\date{}

\begin{document}

\maketitle

\begin{spacing}{1.5}

\section{Intro}

\begin{itemize*}
	\item \textbf{Parallel Processing}: simultaneous use multiple processing elements to solve problem fast
	\item \textbf{Processing Elements (PE)}:
		\begin{itemize*}
			\item Single Processor, multiple Cores
			\item Simple Computer, multiple processors
			\item Many computers, connected via network
			\item Combination of above 
		\end{itemize*}
	\item Problem need to be \textbf{partitioned} into sufficient \textbf{independent parts} for execution on \textbf{parallel PE's}
\end{itemize*}

\section{Serial Computing}

\includegraphics[scale=0.7]{serial-computing.png}

\begin{itemize*}
	\item $\textbf{Problem} \Rightarrow \textbf{discrete series of instructions}$
	\item Executed in series (1 at a time)
\end{itemize*}

\section{Parallel Computing}

\includegraphics[scale=0.7]{parallel-computing.png}

\begin{itemize*}
	\item $\textbf{Problem} \Rightarrow \textbf{Discrete tasks}$
	\item Solved concurrently
	\item Instructions of each part execute in parallel on different PEs
\end{itemize*}

\section{Von Neumann Computation Model}

\includegraphics[scale=0.7]{von-neumann-comp-model.png}

\section{Why Parallel Computing}

Exploits large number of PEs that communicate and coorperate to \textbf{solve large problems fast}

\textbf{Primary reasons}

\begin{itemize*}
	\item Overcome limits of serial computing
	\item Save (wall clock) time
	\item Solve large problems
\end{itemize*}

\textbf{Secondary reasons}

\begin{itemize*}
	\item Take advantage of non-local 
	\item Cost savings - use multiple cheaper commoditized computing resources
	\item Overcome memory constraints
\end{itemize*}

\section{How?}

\begin{itemize*}
	\item Work hard (serial)
	\item Work smart (optimize/algorithms)
	\item Get help (go parallel) 
\end{itemize*}

\section{Parallel Computing Basics}

\includegraphics[scale=0.7]{problem-decomposition.png}

\subsection{Decomposition}

\begin{itemize*}
	\item \textbf{Potential parallelism} in problem dictates how it should be split
	\item \textbf{Granulatity}: size of tasks
\end{itemize*}

\subsection{Tasks}

\begin{itemize*}
	\item \textbf{Scheduling}: tasks $\Rightarrow$ processes/threads
	\item \textbf{Mapping}: processess/threads $\Rightarrow$ cores/processors
\end{itemize*}

\subsection{Dependencies \& Coordination}

\begin{itemize*}
	\item Dependencies constrain scheduling
	\item To execute correctly, syncronization \& coordination needed
\end{itemize*}

\subsection{Performance}

\begin{itemize*}
	\item Throughput vs time
	\item Parallel execution time = Computation time + (data exchange \& scynronization time)
	
\end{itemize*}

\subsection{Challenges}

\begin{itemize*}
	\item Optimize execution resources (less idle)
	\item How to share memory correctly
	\item Automate extraction of parallelism: compilers, language tooling
	\item Verification of correctness
	\item Debugging
	\item Performance monitoring 
\end{itemize*}

\end{spacing}

\end{document}
