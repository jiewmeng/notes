\documentclass[12pt]{article}
\usepackage[a4paper,margin=0.5cm]{geometry}
\usepackage{mdframed}
\usepackage{color}
\usepackage{hyperref}
\usepackage[usenames,dvipsnames,svgnames,table]{xcolor}
\newcommand{\hilight}[1]{\colorbox{Dandelion}{#1}}

\begin{document}

\title{PC1221 - Data Analysis}
\author{Lim Jiew Meng}
\maketitle

\section{Expressing Uncertainty}

\begin{mdframed}
$$
\textrm{best estimated} \pm \textrm{uncertainty} \textrm{ [units]} \\ = x_{best} \pm \delta x \textrm{ [units]}
$$
\end{mdframed}
\vspace{0.5cm}

Example: $1.023 \pm 0.006 m/s^2$

\begin{itemize}
	\item round \textcolor{blue}{uncertainty} to \textcolor{purple}{1 significant figure}
	\item round \textcolor{Mulberry}{measurement} to same precision as \textcolor{blue}{uncertainty}
\end{itemize}

\subsection{Absolute \& Fractional}

\subsubsection{Absolute Uncertainty ($\delta x$)}

Actual amount by which best estimate is uncertain

\subsubsection{Fractional Uncertainty (\large{$\frac{\delta x}{x_{best}}$})}

Significance of uncertainty wrt. best estimated value ($x_{best}$)

\subsection{Accuracy vs Precision}

\subsubsection{Accuracy}

Measure of closeness of \hilight{experimental} values to \hilight{true value}

\subsubsection{Precision}

Measure of closeness of \hilight{repeated measurements}

\subsection{Types}

\subsubsection{Random}

\begin{itemize}
	\item From unknown and unpredictable variations
	\item Reduced by taking several measurements
\end{itemize}

\subsubsection{Systematic}

\begin{itemize}
	\item Associated with measurement instrument or technique
	\item Nearly of same sign/direction and magnitude between measurements
	\item Cannot be eliminated statistically
\end{itemize}

\section{Significant Figures}

\begin{itemize}
	\item All digits significant \hilight{except} those whose sole purpose is to \hilight{determine location of decimal place}
	\item {\bf Add/Subtract}: Use least precise term (decimal point)
	\item {\bf Multiply/Divide}: Use least significant digits
\end{itemize}

\section{Experimental Value vs Accepted Value}

Let $E \pm \delta E$ be experimental value with uncertainty, $A$ be accepted value. If $A$ lies between $E \pm \delta E$, experimental value $=$ accepted value within uncertainty

\vspace{0.5cm}

{\bf Percentage discrepancy} quantifies accuracy of measurement 

\vspace{0.5cm}

\begin{mdframed}
	$$\textrm{Percentage Discrepancy} = \frac{|E-A|}{A} \times 100\%$$
\end{mdframed} 

\subsection{Compare 2 experimental values}

2 experimental values are equal within uncertainty, as long as theres some overlap between $E_1 \pm \delta E_1$ and $E_2 \pm \delta E_2$

$$\textrm{Percentage Difference} = \frac{|E_1 - E_2|}{|E_1 + E_2|/2} \times 100\%$$

\section{Mean/Average/Best Estimate}

\begin{mdframed}
	$$\overline{x} = \frac{1}{N} \sum_{i=1}^{N} x_i$$
\end{mdframed}

\section{Standard Deviation ($\sigma$)}

Quantifies \hilight{spread of data about mean}. Used to compare \hilight{individual value} wrt. \hilight{mean}

\begin{itemize}
	\item 68.3\% of data lies within $1 \sigma$
	\item 95.5\% of data lies within $2 \sigma$
	\item 99.73\% of data lies within $3 \sigma$
\end{itemize}

\section{Standard Error ($\alpha$)}

Measures \hilight{spread of all means about overall mean}. Used to compare \hilight{mean} wrt. \hilight{true value}

\vspace{0.5cm}

\begin{mdframed}
$$\alpha = \frac{\sigma}{\sqrt{N}}$$
\end{mdframed}

\section{Uncertainty Propagation}

\subsection{Add/Subtract}

\begin{mdframed}
	$$\delta X = \sqrt{(\delta A)^2 + (\delta B)^2}$$
\end{mdframed}

\subsection{Multiply/Divide}

\begin{mdframed}
	$$\delta X = X \sqrt{(\frac{\delta A}{A})^2 + (\frac{\delta B}{B})^2}$$
\end{mdframed}

\section{Linear Least Squares Fit}

Using excel's \texttt{LINEST()}. \href{http://www.youtube.com/watch?v=ECA2VSOhbuU}{YouTube Video}

\end{document}