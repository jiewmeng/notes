\documentclass[12pt]{article}

\usepackage[T1]{fontenc}
\usepackage{setspace}
\usepackage[a4paper, margin=1.2cm, bottom=2.5cm]{geometry}
\usepackage{graphicx}
\usepackage{mdwlist}
\usepackage{amsmath}

\graphicspath{{./_img/}}
\setlength{\parindent}{0cm}

\title{\textbf{Rotation of Rigid Object about Fixed Axis}}
\date{}

\begin{document}

\maketitle

\begin{spacing}{1.5}

\section{Radians}

$$1 \textrm{rad} = \frac{360}{2\pi}$$

$$\theta \textrm{ [rad]} = \frac{\pi}{180} \theta \textrm{ [degrees]}$$

\section{Arc Length}

$$s = r \theta$$

\section{Angular stuff ...}

\subsection{Position}

$\theta$ measured from the +ve $x$ axis, counter clock wise

\subsection{Displacement (degrees)}

$$\Delta \theta$$

\subsection{Speed (rad/s)}

$$\overline{\omega} = \frac{\Delta \theta}{\Delta t}$$

$$\omega = \lim_{\Delta t \to 0} \frac{\Delta \theta}{\Delta t}$$

\subsection{Acceleration (rad/$s^2$)}

$$\overline{\alpha} = \frac{\Delta \omega}{\Delta t}$$

$$\alpha = \lim_{\Delta t \to 0} \frac{\Delta \omega}{\Delta t}$$

\subsection{Directions (which is $+$ve)}

Using right hand rule


\section{Angular \& Linear Quantities}

\begin{tabular}{p{5cm} | l}
	
	$\omega_f = \omega_i + \alpha t$ &
	$v_f = v_i + at$ \\
	
	$\theta_f = \theta_i + \omega_i t + \frac{1}{2} \alpha t^2$ & 
	$x_f = x_i + v_i t + \frac{1}{2} at^2$ \\
	
	$\omega_f^2 = \omega_i^2 + 2\alpha(\theta_f - \theta_i)$ &
	$v_f^2 = v_i^2 + \frac{1}{2} a (x_f - x_i)$ \\
	
	$\theta_f = \theta_i + \frac{1}{2} (\omega_i + \omega_f) t$ & 
	$x_f = x_i + \frac{1}{2} (v_i + v_f) t$
	
\end{tabular}

$$s = r \theta$$

$$v = r \omega$$

$$a = r \alpha$$

$$a_c = r \omega^2$$

\section{Moment of Inertia}

$$I = mr^2$$

Smaller $\rightarrow$ easier to rotate

\section{Rotational Kinetic Energy ($J$)}

$$K_R = \frac{1}{2} I \omega^2$$

\section{Parallel Axis Theorm}

If axis is not though center of mass, but some axis parallel to axis through center of mass

$$I = I_{CM} + MD^2$$

Where $I_{CM}$ is axis through center of mass. $D$ is distance from center of mass (rmb: $d = \sqrt{x^2 + y^2}$)

\section{Torque (Rotational Force, $N \cdot m$ (not joules))}

$$\tau = (F \sin{\theta}) \cdot d = I \alpha$$

$d$ refers to moment arm. Counterclockwise positive

\section{Power}

$$P_R = \tau \omega$$

\section{Work-KE Theorem in Rotational Motion}

$$\sum W = \int_{w_i}^{w_f} I \omega \; d \omega = \frac{1}{2} I \omega_f^2 - \frac{1}{2} I \omega_i^2$$

\section{Equations}

\includegraphics[scale=0.7]{rotational-equ.png}

\section{Pure Rolling}

Velocity at center of mass, $v_{CM}$

$$v_{CM} = R \omega$$

Acceleration at center of mass, $a_{CM}$

$$a_{CM} = R \alpha$$

\includegraphics[scale=0.7]{pure-rotation-combi.png}

\section{Total KE of rolling object}

$$KE = \frac{1}{2} mv_{cm}^2 + \frac{1}{2} I_{CM} \omega^2$$

\end{spacing}

\end{document}
