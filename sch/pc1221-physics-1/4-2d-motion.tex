\documentclass[12pt]{article}

\usepackage[T1]{fontenc}
\usepackage{setspace}
\usepackage[a4paper, margin=1.2cm, bottom=2.5cm]{geometry}
\usepackage{graphicx}
\usepackage{mdwlist}
\usepackage{amsmath}

\graphicspath{{./_img/}}
\setlength{\parindent}{0cm}

\title{\textbf{PC1221 - 4 - 2D Motion}}
\date{}

\begin{document}

\maketitle

\begin{spacing}{1.5}

\section{Displacements}

\begin{center}
\boxed{From:
$$s_f = s_i + v_i t + \frac{1}{2} at^2$$}
\end{center}

$$x_f = v_{xi} t = (v_i \cos{\theta})t$$

$$y_f = v_{yi} t + \frac{1}{2} a_y t^2 = (v_i \sin{\theta}) t - \frac{1}{2} gt^2$$

$$y = (\tan{\theta})x - (\frac{g}{2v_i^2 \cos^2{\theta}})x^2$$

\section{Max Height \& Range}

Only for symetric motion

$$h = \frac{v_i^2 \sin^2{\theta}}{2g}$$

$$R = \frac{v_i^2 \sin{2\theta}}{g}$$

\textbf{Proof (h)}

$$\begin{aligned}
x &= (v_i \cos{\theta})t \\
t &= \frac{x}{v_i \cos{\theta}}
\end{aligned}$$

$$y = (v_i \sin{\theta})t - \frac{1}{2} gt^2$$

$$y = (v_i \sin{\theta}) \times \frac{x}{v_i \cos{\theta}} - \frac{1}{2} g (\frac{x}{v_i \cos{\theta}})^2 $$

Max when $v = 0$, just before coming down

$$v_f = v_i \sin{\theta} - gt = 0$$

$$v_i \sin{\theta} - gt = 0$$

$$t = \frac{v_i \sin{\theta}}{g}$$

Sub into $y$

$$y = h = v_i \sin{\theta} (\frac{v_i\sin{\theta}}{g}) - \frac{1}{2} g (\frac{v_i \sin{\theta}}{g})^2$$

$$h = \frac{v_i^2 \sin^2{\theta}}{g} - \frac{v_i^2 \sin^2{\theta}}{2g} = \frac{v_i^2 \sin^2{\theta}}{2g}$$

\textbf{Max Range}

Let $y = 0$ (on ground)

$$v_i (\sin{\theta} )t - \frac{1}{2} gt^2 = 0$$

$$t = \frac{2 v_i \sin{\theta}}{g}$$

$$R = (v_i \cos{\theta}) \times \frac{2 v_i \sin{\theta}}{g} = \frac{v_i^2 \sin{(2\theta)}}{g}$$

\section{Uniform Circular Motion}

\begin{itemize*}
	\item When object move along circular path with constant speed
	\item Change in velocity related to acceleration
	\item velocity vector tangent to path of object
\end{itemize*}

$$v_f = v_i + \Delta v$$

\subsection{Centripetal Acceleration}

\begin{itemize*}
	\item Acceleration perpendicular to path of motion
	\item Points to center of circle
\end{itemize*}

$$a_c = \frac{c^2}{r}$$

\subsection{Period}

\begin{itemize*}
	\item Time require for 1 cycle
	\item Distance = Circumference
\end{itemize*}

$$T = \frac{2 \pi r}{v}$$

\subsection{Tangential Acceleration}

\includegraphics[scale=0.6]{tangential-acceleration.png}

Causes change in velocity (magnitude). 

Tangential Acceleration: $a_t = \frac{\Delta v}{\Delta t}$

Radial Acceleration: $a_r = a_c = \frac{v^2}{r}$

$a_{total} = \sqrt{a_r^2 + a_t^2}$

\section{Relative Velocity}

\begin{itemize*}
	\item $r$ is position seen from stationary frame
	\item $r'$ is position from moving frame 
\end{itemize*}

$$r' = r - v_0 t$$

\subsection{Acceleration from different frame of reference}

Acceleration of particle measure from different frames of reference (moving at constant velocity) will be the same as: $v_0$ is a constant, thus having no effect on change of velocity

$v' = v - v_0$

\end{spacing}

\end{document}
