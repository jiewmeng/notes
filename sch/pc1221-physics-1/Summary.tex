\documentclass[12pt, twocolumn]{article}

\usepackage[T1]{fontenc}
\usepackage{setspace}
\usepackage[a4paper, margin=1.2cm, bottom=2.5cm]{geometry}
\usepackage{graphicx}
\usepackage{mdwlist}
\usepackage{amsmath, amssymb}
\usepackage{hyperref}
\usepackage{color, soul}
\usepackage{pifont}
\usepackage{tabularx}

\newcommand{\tick}{\ding{52}}
\newcommand{\cross}{\ding{54}}

\graphicspath{{./_img/}}
\setlength{\parindent}{0cm}

\title{\textbf{Physics 1 Final Notes}}
\date{}

\begin{document}

\maketitle

\begin{spacing}{1.5}

\section{Kinematics}

\begin{tabular}{p{4cm}|p{1.3cm}|p{2cm}}

& \textbf{Avg.} & \textbf{Inst.} \\
\hline
\textbf{Displacement ($m$)} & $\Delta x$ & - \\

\textbf{Velocity ($m/s$)} & $\dfrac{\Delta x}{\Delta t}$ & $\lim_{\Delta t \to 0} \dfrac{\Delta x}{\Delta t}$ \\[0.3cm]

\textbf{Acceleration ($m/s^2$)} & $\dfrac{\Delta v}{\Delta t}$ & $\lim_{\Delta t \to 0} \dfrac{\Delta v}{\Delta t}$ \\[0.3cm]

\end{tabular}

$$v_f = v_i + at$$
$$s_f = s_i + \frac{1}{2} (v_i + v_f) t$$
$$s_f = s_i + v_i t + \frac{1}{2} at^2$$
$$v^2 = v_i^2 + 2a(x_f - x_i)$$

\section{2D Motion}

$$h = \frac{v_i^2 \sin^2{\theta}}{2g}$$

$$R = \frac{v_i^2 \sin{2\theta}}{g}$$

\subsection{Uniform Circular Motion}

When object moves along circular path with constant speed
$$a_c = \frac{c^2}{r}$$

\subsubsection{Period (time for 1 cycle)}

$$T = \frac{d}{v} = \frac{2 \pi r}{v}$$

\subsubsection{Tangental Acceleration}

Causes change in velocity

$$a_t = \frac{\Delta v}{\Delta t}$$
$$a_r = a_c$$
$$a_{tot} = \sqrt{a_r^2 + a_t^2}$$

\section{Force}

\subsection{Newton's 1st Law}

When net force = 0, object at rest will stay at rest. Object in motion will continue moving at constant velocity

\begin{itemize*}
	\item Inertia: tendency of object to resist change to velocity
	\item Mass ($kg$): Independent of surroundings 
	\item Weight ($N$): Dependent on gravity. $W = mg$
\end{itemize*}

\subsection{Newton's 2nd Law}

$$F = ma$$

\subsection{Newton's 3rd Law: Equal \& opposite force exists}

\begin{itemize*}
	\item Forces occurs in pairs
	\item |Action Force| = |Reaction/Normal Force|
\end{itemize*}

\subsection{Friction}

\textbf{Static friction} keeps object from moving
$$f_s \le \mu_s n$$

\textbf{Kinetic friction} when object is moving
$$f_k = \mu_k n$$

\section{Circular Motion}

$$F_{circ} = m \frac{v^2}{r}$$

\subsection{Conical Pendulum}

$$v = \sqrt{Lg \sin{\theta} \tan{\theta}}$$

Where $L$ is length of string

\subsection{Motion on Horizontal Cicle}

$$T = m \frac{v^2}{r} \qquad \qquad 
v = \sqrt{\frac{Tr}{m}}$$

\subsection{Horizontal Curve}

Static friction provides centripetal force

$$\mu_s \underbrace{m g}_{\text{n}} = m \frac{v^2}{r} \qquad \Rightarrow \qquad 
v = \sqrt{\mu_s gr}$$ 

\subsection{Banked Curve}

$$v = \sqrt{rg \tan{\theta}}$$

\subsection{Vertical Circle}

Note the direction of centripetal force

$$F_y = m \frac{\pm v^2}{r} = n \pm mg$$

\section{Energy ($\approx Work$)}

\subsection{Work ($J = Nm$)}

$$W = F \Delta x = \int F \; dx$$

\subsection{Hooke's Law/Spring force}

$$F_{spring} = -kx$$
$$PE_{spring} = \frac{1}{2} kx^2$$

\subsection{Kinetic energy}

$$KE = \frac{1}{2} mv^2$$

\subsection{Potential energy}

$$GPE = mgh$$

\subsection{Power ($W = J/s$)}

$$P = \frac{W}{\Delta t} = Fv$$

\subsection{Conservation of Mechanical Energy}

$$E_{mech} = KE + GPE$$
$$E_{initial} = E_{final}$$

\subsubsection{Non-conservative forces}

\begin{align*}
\Delta E_{mech} &= \Delta KE + \Delta PE \\
&= -f_k d \\
&= - \mu_k n d
\end{align*}

\section{Momentum \& Impulse}

\subsection{Momentum}

$$p = mv$$
$$F = \frac{d}{dx} \; p$$

\subsubsection{Conservation of momentum}

$$\underbrace{p_{1i} + p_{2i}}_{\text{initial momentum}} = \underbrace{p_{1i} + p_{2i}}_{\text{final momentum}}$$

\subsection{Impulse}

Change in momentum
$$J = \Delta p$$

\subsection{Collisions}

\begin{tabular}{l | cc | l}
	\textbf{Conserved?} & \textbf{KE} & \textbf{$p$}& \textbf{Perfect} \\
	\hline
	\textbf{Inelastic} & \cross & \tick & Stick together \\
	\textbf{Elastic} & \tick & \tick & Atomic level only \\
\end{tabular}

\subsubsection{(Perfectly) Inelastic}

Conservation of momentum
$$m_1 v_{1i} + m_2 v_{2i} = \underbrace{(m_1 + m_2)}_{\text{stick tgt}} v_f$$

\subsubsection{Elastic}

Conservation of momentum
$$m_1 v_{1i} + m_2 v_{2i} = m_1 v_{1f} + m_2 v_{2f}$$

Conservation of KE
$$\frac{1}{2} m_1 v_{1i}^2 + \frac{1}{2} m_2 v_{2i}^2 = \frac{1}{2} m_1 v_{1f}^2 + \frac{1}{2} m_2 v_{2f}^2$$

\section{Rotational Kinematics}

$$1 \; rad = \frac{360}{2 \pi}$$
$$\theta \; [rad] = \frac{180}{\pi} \; \theta \; [degrees]$$

\vspace{0.5cm}

\begin{tabular}{p{3.5cm}  p{2cm}}

\textbf{Arc Length ($s$)} & $r \theta$ \\[0.1cm]

\textbf{Displacement} & $\Delta \theta$ \\[0.3cm]

\textbf{Speed ($\overline{\omega}$)} & $\dfrac{\Delta \theta}{\Delta t}$ \\[0.5cm]

\textbf{Acceleration ($\overline{\alpha}$)} & $\dfrac{\Delta \omega}{\Delta t}$ \\[0.3cm]

\end{tabular}

\vspace{0.5cm}

Rotational kinematic equations are similar to (linear) kinematic equations. eg. 
$$\omega_f = \omega_i + \alpha t \qquad \qquad 
v_f = v_i + at$$

\subsection{Linear kinematics to rotational}

$$s = r \theta$$
$$v = r \omega$$
$$a = r \alpha$$
$$a_c = r \omega^2$$

\section{Thermodynamics}

\subsection{Linear Expansion}

$$\Delta L = \alpha L_i \Delta T$$

\subsection{Area Expansion}

$$\Delta A = 2\alpha A_i \Delta T$$

\subsection{Volume Expansion}

$$\Delta V = \beta V_i \Delta T$$

For a solid, $\beta = 3\alpha$

\subsection{Gas Formula}

$$PV \; [J] = nRT$$

\underline{$T$ is in $K$elvins}. $n$ is number of moles. $R$ is universal gas constant.

\subsection{Specific Heat ($c$)}

How \underline{insensitive} a substance is to \underline{addition of energy}

$$c = \frac{Q}{m \; \Delta t} \qquad \qquad 
Q = c \; m \; \Delta t$$

$Q$ is energy (eg. gained/lost in heating/cooling)

\begin{itemize*}
	\item Higher specific heat $\rightarrow$ worse burn
	\item Lower specific heat $\rightarrow$ faster to cool down
\end{itemize*}

\subsection{Phase change}

Change in state. No change in temperature

\subsection{Latent heat ($L$)}

Energy needed to change state of substance

$$L = \frac{Q}{m} \qquad \qquad Q = \pm \; m L$$

\begin{itemize*}
	\item Positive energy when energy is transfered \underline{into} system (eg. boiling)
	\item Negative energy when energy is transfered \underline{out} of system (eg. freezing)
\end{itemize*}

\subsection{Work}

$$W = \int_{V_i}^{V_f} \; P \; dV$$

* Think area under PV diagram

\begin{itemize*}
	\item Positive work if volume increases (arrow leftwards on PV)
	\item Negative work if volume decreases (arrow right on PV)
\end{itemize*}

\subsection{1st Law of Thermodynamics}

$$\Delta E_{int} = Q + W$$

\begin{itemize*}
	\item $\Delta E_{int}$ : Change in internal temperature of system
	\item $Q$ : Heat energy added into system
	\item $W$ : Work done \underline{on} the system
\end{itemize*}

\subsubsection{Isolated System/Adiabatic Free Expansion}

$$Q = W = 0 \qquad \Rightarrow \qquad \Delta E_{int} = 0$$

Example: air filling vaccum

\subsubsection{Cyclic Process}

$$\Delta E_{int} = 0 \qquad \Rightarrow \qquad Q = -W$$

\subsubsection{Adiabatic Process}

No energy transfer with surroundings. By insulation or fast process
$$Q = 0 \qquad \Rightarrow \qquad \Delta E_{int} = W$$

\subsubsection{Isothermal Process}

$$\Delta E_{int} = 0 \qquad \Rightarrow \qquad Q = -W$$

\subsubsection{Isobaric Process}

$$W = -P \; \Delta V$$

\subsubsection{Isovolumetric Process}

$$W = 0 \qquad \Rightarrow \qquad Q = \Delta E_{int}$$

\end{spacing}

\end{document}
