\documentclass[12pt]{article}

\usepackage[T1]{fontenc}
\usepackage{setspace}
\usepackage[a4paper, margin=1.2cm, bottom=2.5cm]{geometry}
\usepackage{graphicx}
\usepackage{mdwlist}
\usepackage{amsmath, amssymb}
\usepackage{hyperref}
\usepackage{color, soul}

\graphicspath{{./_img/}}
\setlength{\parindent}{0cm}

\title{\textbf{CS1231 - Cardinality}}
\date{}

\begin{document}

\maketitle

\begin{spacing}{1.5}

\section{Cardinality}

\begin{itemize*}
	\item Count of elements in set
	\item \textbf{\hl{Bijection $\Rightarrow$ Same cardinality}}
		\begin{enumerate*}
			\item (Injective) $f(m) = f(n) \Rightarrow m = n$
			\item (Surjective) $\forall y \in Y$, show that $\exists x \in X$ such that $f(x) = y$
		\end{enumerate*}
\end{itemize*}

\subsection{Cardinality Relation}

Is equivalence relation 

\begin{itemize*}
	\item Reflexive: $|A| = |A|$
	\item Symmetric: $|A| = |B| \Rightarrow |B| = |A|$
	\item Transitive: $(|A| = |B| \wedge |B| = |C|) \Rightarrow |A| = |C|$
\end{itemize*}

\section{Countably infinite}

\begin{itemize*}
	\item Set is countably infinite if $|S| = \mathcal{N}_0 = |Z^+|$

	\item If a set can be \textbf{listed}, its countably infinite (prop. 9.3.4)

	\item If $A$ and $B$ is countably infinite, then so is $A \times B$ (since they can be listed)

	\item (Generalized) The cartesian product of many countably infinite sets is also countably infinite (prop. 9.3.6)
	
	\item Union of countably many countable sets is countable (prop. 9.3.7)
	
	\item Subset of a countable set is countable
	
	\item Superset of an uncountable set is uncountable
\end{itemize*}

\subsection{$\mathbb{Q}$ is countable}

\begin{enumerate*}
	\item $\mathbb{Q}^+$, $\frac{a}{b}$, can be expressed as $<a, b>$. Where $a, b \in \mathbb{Z}^+$ and $a \perp b$ (unique representation of $\mathbb{Q}^+$)
	\item $\mathbb{Q}^+ \subset \mathbb{Z^+} \times \mathbb{Z^+}$
	\item Therefore, $\mathbb{Q}^+$ is countable (subset of a countable set)
	\item $\mathbb{Q}^-$, similar to $\mathbb{Q}^+$, can be expressed as $<-m, n>$
	\item Theres a bijection $f(<-m, n>) = <m, n>$ thus $\mathbb{Q}^-$ is countable (same cardinality if theres a bijection)
	\item ${<0, 1>}$ is countable (1 element)
	\item $\mathbb{Q}^+ \cup \mathbb{Q}^- \cup <0, 1>$ is countable (union of countable sets is countable)
\end{enumerate*}

\subsection{$\mathbb{R}$ is uncountable}

Cantor's argument

\subsection{Injection but no surjection $\rightarrow$ larger cardinality}

If theres an injection $f : A \rightarrow B$ but no surjection, then $|A| < |B|$

\subsection{Cardinality of power set larger than set}

$$|A| < |\mathcal{P}(A)|$$

\subsection{(Theroem 9.4.6)}

$$|\mathbb{R}| = |\mathcal{P}(\mathbb{Z}^+)|$$

\end{spacing}

\end{document}
