\documentclass[12pt]{article}

\usepackage[T1]{fontenc}
\usepackage{setspace}
\usepackage[a4paper, margin=1.2cm, bottom=2.5cm]{geometry}
\usepackage{graphicx}
\usepackage{mdwlist}
\usepackage{amsmath, amssymb}
\usepackage{hyperref}
\usepackage{color, soul}

\graphicspath{{./_img/}}
\setlength{\parindent}{0cm}

\title{\textbf{Number Theory (\& Systems)}}
\date{}

\begin{document}

\maketitle

\begin{spacing}{1.5}

\section{Number Theory}

\subsection{Natural Numbers ($\mathbb{N}$)}

Smallest set such that 

\begin{enumerate*}
	\item $\exists 0 (0 \in \mathbb{N})$
	\item There exists a successor function $s$ on $\mathbb{N}$. $s(n)$ (denoted $n'$) is the successor of $n$. (Successor is $n+1$)
	\item $\forall n \in \mathbb{N} (n' \neq 0)$
	\item $\forall n, m \in \mathbb{N} (n' = m' \Rightarrow n = m)$
	\item $\forall K \subset \mathbb{N} \forall n \in \mathbb{N} ((0 \in K \wedge (n \in K \Rightarrow n' \in K)) \Rightarrow K = \mathbb{N})$
\end{enumerate*}

\subsubsection{Less than equals ($\le$)}

$$
\forall n, m \in \mathbb{N}_c
(n \le m \Leftrightarrow n \subset m)
$$

\underline{\textbf{$(\mathbb{N}_c, \le)$ is a partial order}}

\begin{enumerate*}
	\item We prove $(\mathbb{N}_c, \le)$ is a pre-order
		\begin{enumerate*}
			\item $(\mathbb{N}_c, \le)$ is reflexive by reflecivity of $\subset$
			\item $(\mathbb{N}_c, \le)$ is transitive by transitivity of $\subset$
			\item Therefore a pre-order
		\end{enumerate*}
	\item $(\mathbb{N}_c, \le)$ is anti-symmetric by anti-symmetry of $\subset$
	\item Therefore a partial order
\end{enumerate*}

\underline{\textbf{Ordering Lemma}}

$$n \le m \vee m \le n$$

\underline{\textbf{$(\mathbb{N}_c, \le)$ is a total order}}

\begin{enumerate*}
	\item We know $(\mathbb{N}_c, \le)$ is a partial order
	\item We know any 2 distinct elements in $\mathbb{N}_c$ are comparable by ordering lemma
	\item Therefore total order
\end{enumerate*}

\underline{\textbf{$(\mathbb{N}_c, \le)$ is a well ordered}}

\begin{enumerate*}
	\item Sketch: prove every subset of $\mathbb{N}_c$ has a smallest element
\end{enumerate*}

\subsubsection{Addition}

\begin{itemize*}
	\item $\forall n \in \mathbb{N} (n+0=n)$
	\item $\forall n, m \in \mathbb{N} (n+m' = (n+m)')$
\end{itemize*}

\subsubsection{Multiplication}

\begin{itemize*}
	\item $\forall n \in \mathbb{N} (n \times 0 = 0)$
	\item $\forall n, m \in \mathbb{N} (n \times m' = (n \times m) + n)$
\end{itemize*}

\subsection{Natural Numbers}

Let $\approx$ be a relation on $\mathbb{N} \times \mathbb{N}$ such that

$$\forall n_1, n_2, m_1, m_2 (<n_1, n_2> \approx <m_1, m_2> \Leftrightarrow n_2 + m_1 = m_2 + n_1)$$
$$\mathbb{Z} = (\mathbb{N} \times \mathbb{N})/\approx$$

$n$ if $<0, n>$. $-n$ if $<n, 0>$

\subsubsection{Addition}

$$<a_1, a_2> + <b_1, b_2> = < a_1 + b_1, a_2 + b_2 >$$

\subsubsection{Subtraction}

$$<a_1, a_2> - <b_1, b_2> = < a_1 + b_2, a_2 + b_1 >$$

\subsubsection{Multiplication}

$$<a_1, a_2> \times <b_1, b_2> = < (a_1 \times b_2) + (a_2 \times b_1), (a_1 \times b_1) + (a_2 \times b_2) >$$

\subsection{Rational Numbers}

$$\forall n_1, m_1 \in \mathbb{Z} \forall n_2, m_2 \in (\mathbb{Z} \textbackslash \{0\})$$
$$(<n_1, n_2> \approx <m_1, m_2> \Leftrightarrow n_2 \times m_1 = m_2 \times n_1)$$
$$\mathbb{Q} = (\mathbb{Z} \times (\mathbb{Z} \textbackslash {0})) / \approx$$

\subsubsection{Multiplication}

$$a \times b = < a_1 \times b_1, a_2 \times b_2 > = \frac{a_1\times b_1}{a_2 \times b_2}$$

\subsubsection{Addition}

$$a + b = < (a_1 \times b_2) + (b_1 \times b_2), a_2 \times b_2 >$$

\subsubsection{Subtraction}

$$a - b = < (a_1 \times b_2) - (b_2 \times a_2), a_2 \times b_2 >$$

\section{Divisibility}

$m$ divides $n$, denoted \boxed{\underbrace{m}_{\text{divisor}} \; | \; \; n}

$\exists q \in \mathbb{N} (n = m \times q)$ 

\subsection{Proposition 8.2.2}

$$m \; | \; n \Rightarrow m \le n$$

\subsection{Proposition 8.2.3 - Remainder less than divisor}

$$n \in \mathbb{N} \; 
m \in \mathbb{N}^+ \; 
((n = q \times m + r) \wedge (r < m))$$

\subsection{Division Algorithm (Proposition 8.2.4)}

$$n \in \mathbb{N} \;
m \in \mathbb{N}^+ \;
\exists ! q, \; r \in \mathbb{N} \; 
(n = q \times m + r \wedge r < m)$$

\begin{itemize*}
	\item $n$ : dividend
	\item $m$ : divisor
	\item $r$ : remainder OR modulo $m$ of $n$
	\item $q$ : quotient 
\end{itemize*}

\section{Co-prime}

$n$ and $m$ are \textbf{relatively prime/co-prime}, denoted \boxed{n \perp m} iff

$$\forall c \in \mathbb{N}^+ \; 
(
((c | n) \wedge (c | m)) \Rightarrow c = 1
)$$ 

\section{Prime numbers}

$$p > 1 \wedge (\forall n \in \mathbb{N}^+ \; (n \; | \; p \Rightarrow (n = p \vee n = 1)))$$

\subsection{Composite (not prime)}

$$\neg \text{prime}(n) \wedge n \neq 1$$

\subsubsection{A composite number can be expressed as a multiple of 2 $\mathbb{N}^+$ (Proposition 8.2.9)}

$$\exists n, m \in \mathbb{N}^+ \; ((1 < n < q) \wedge (1 < m < q) \wedge q = n \times m) $$

\section{GCD (Proposition 8.2.10)}

There exists a unique number $c \in \mathbb{N}^+$ such that

$$
\underbrace{(c \; | \; n) \wedge (c \; | \; m)}_{\textrm{common divisor}} \wedge 
\underbrace{(\forall q \in \mathbb{N}^+ (((q | n) \wedge (q | m)) \Rightarrow q \le c))}_{\text{largest unique}}$$

\subsection{GCD of co-prime numbers is 1 (Proposition 8.2.12)}

$$n \perp m \Leftrightarrow gcd(n, m) = 1$$

\subsection{Bezout Identity (Proposition 8.2.13)}

$$n, m \in \mathbb{N}^+ \qquad \exists a, b \in \mathbb{Z} \; (n \times a + m \times b = gcd(m, n))$$

\subsection{Euclid's Lemma (Prop. 8.2.15)}

$$n,m,p \in \mathbb{N}^+ \qquad 
(\text{prime}(p) \wedge (p \; | \; n \times m)) \Rightarrow (p | n \vee p | m)$$

\section{Factorization}

Factorization of $n$ is a collection of possible duplicate prime numbers, $p_i$, such that

$$n = \Pi_{i \in I} p_i$$

\section{Fundamental Theorem of Arithmetic}

Every $\mathbb{N}^+$ has a unique factorization

\subsection{Common divisor divides the GCD (Prop. 8.3.4)}
$$\forall q \in \mathbb{N}^+ \; (((q | n) \wedge (q | m)) \Rightarrow q | gcd(m, n))$$

\subsection{LCM}

\subsubsection{(Prop. 8.3.5)}

$$n, m \in \mathbb{N}^+ \qquad 
(n | c) \wedge (m | c) \wedge
(\forall q \in \mathbb{N}^+ \; (((n|q) \wedge (m|q)) \Rightarrow c \le q))$$


\subsubsection{(Prop. 8.3.6)}

$$(n | lcm(n, m)) \wedge (m | lcm(n, m)) \wedge 
(\forall q \in \mathbb{N}^+ \; ((n|q) \wedge (m|q) \Rightarrow lcm(n, m) \le q))$$


\subsubsection{LCD (Prop. 8.3.7)}

$$\forall q \in \mathbb{N}^+ \; ((n|q) \wedge (m|q)) \Rightarrow lcd(n, m) | q$$

\section{Modular Arithmetic}

$n$ is congruent to (m modulo c), denoted $n \equiv m \;  (\text{mod } c)$ iff

$$(m < n \wedge c | n-m) \vee (n < m \wedge c | m - n) \vee (n = m)$$

\subsection{(Prop. 8.4.2)}

$$n \equiv m \; (\text{mod } c) \Leftrightarrow (n \text{ mod } c = m \text{ mod } c)$$

\subsection{Congruence relation}

$$< n, m > \in \equiv_{\text{mod } c} \text{ iff } \; n \equiv_{\text{mod } c} m$$

\subsection{(Prop. 8.4.5)}

$$(n+m) \text{ mod } c = (n \text{ mod } c + m \text{ mod } c) \text{ mod } c$$

\subsection{(Prop. 8.4.6)}

$$(n \times m) \text{ mod } c = (n \text{ mod } c \times m \text{ mod } c) \text{ mod } c$$

\subsection{(Prop. 8.4.7)}

$$(n_1 \equiv m_1 (\text{mod } c) \wedge 
n_2 \equiv m_2 (\text{mod } c)) \Rightarrow
( n_1 + n_2 \equiv (m_1 + m_2) (\text{mod } c) )$$


\subsection{(Prop. 8.4.8)}

$$(n_1 \equiv m_1 (\text{mod } c) \wedge 
n_2 \equiv m_2 (\text{mod } c)) \Rightarrow
( n_1 \times n_2 \equiv (m_1 \times m_2) (\text{mod } c) )$$


\subsection{Fermat's Little Theorem}

$$\text{prime}(p) \Rightarrow a^p \equiv a (\text{mod } p)$$

\subsubsection{(Prop 8.4.10)}

$$(\text{prime}(p) \wedge a \perp p) \Rightarrow a^{p-1} \equiv 1 (\text{mod } p)$$

\end{spacing}
\end{document}
