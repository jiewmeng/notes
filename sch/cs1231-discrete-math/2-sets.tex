\documentclass[12pt]{article}

\usepackage[T1]{fontenc}
\usepackage{setspace}
\usepackage[a4paper, margin=1.2cm, bottom=2.5cm]{geometry}
\usepackage{graphicx}
\usepackage{mdwlist}
\usepackage{amsmath, amssymb}
\usepackage{hyperref}

\graphicspath{{./_img/}}
\setlength{\parindent}{0cm}

\title{\textbf{CS1231 - Sets}}
\date{}

\begin{document}

\maketitle

\begin{spacing}{1.5}

\section{Terminology}

\begin{itemize*}
	\item $\in$: Membership
	\item $\subset$: Includes
\end{itemize*}

\section{Axiomatic Set Theory}

\subsection{Empty Set ($\varnothing$ or ${}$)}

Set with no elements

$\underbrace{\exists X}_{\textrm{empty set}} \underbrace{(\forall Y (Y \notin X))}_{\textrm{no elements in set}}$

\subsection{Empty set is a subset of all sets}

$\underbrace{\forall X}_{\textrm{empty set}}  \underbrace{\forall Z}_{\textrm{any set}} ( \underbrace{\forall Y (Y \notin X)}_{\textrm{empty set}} \Rightarrow \underbrace{(X \subset Z)}_{\textrm{subset of any set}})$

\subsection{Extensionality/Equality}

Sets are equal iff the have the same elements

$\forall X \forall Y (\underbrace{(\forall Z (Z \in X \Leftrightarrow Z \in Y))}_{\textrm{same elements}} \Leftrightarrow X = Y)$

$\forall X \forall Y (\underbrace{(X \subset Y \wedge Y \subset X)}_{\textrm{subsets of each other}} \Leftrightarrow X = Y)$

\subsection{Pairing}

A set ($Z$) that contains sets $X$ \& $Y$ exists

$\forall X \forall Y \exists Z \underbrace{\forall T ((T = X \vee T = Y)}_{\textrm{T is either in X or Y}} \Leftrightarrow T \in Z)$

\subsection{Unordered Pair}

Pair of 2 sets eg. $\{ X, Y \}$ is the set that contains $X$ \& $Y$

\subsection{Singleton}

Set $\{X, X\} = \{X\}$ is called a singleton

\subsection{Unions}

If $S$ is a set of sets, the $T$ (the union) exists which contains all elements in a set in S

$\forall S \exists T 
\underbrace{\forall Y ((Y \in T)}_{\textrm{all elems in union}} 
\Leftrightarrow 
\underbrace{\exists Z((Z \in S)}_{\textrm{one of the sets}} \wedge 
\underbrace{(Y \in Z)}_{\textrm{y is in one of the sets}}))$

\subsection{Power Set ($\mathcal{P}(S)$)}

All possible subsets of set

$\forall S \underbrace{\exists T}_{\text{a possible subset}} \underbrace{\forall X ((X \in T)}_{\text{all elem of possible subset}} 
\underbrace{\Leftrightarrow (X \subset S)}_{\text{x an elem of the set}})$

\underline{Examples}: 

$\mathcal{P}(\{1,2\}) = \{ \varnothing, \{1\}, \{2\}, \{1,2\} \}$

\subsection{Regularity/Axiom of Foundation}

Every non empty set has an element disjoint from the set. Also see \href{http://math.stackexchange.com/questions/204865/axiom-of-regularity}{math.stackexchange question}

$\forall X (X \neq \varnothing \Rightarrow (\exists Y (Y \in X \wedge \forall Z (Z \in X \Rightarrow Z \notin Y))))$

\underline{Remember}: everything is a set so ...

$0 = \varnothing$

$1 = \{0\}$

$2 = \{0, 1\}$

$3 = \{0, 1, 2\}$

$4 = \{0, 1, 2, 3\}$ ... 

So 4 has 3 which is not in 2

\subsubsection{No set is a member of itself}

$\forall X (X \notin X)$

\subsection{Infinity Set}

$\exists X (\varnothing \in X \wedge (\forall Y (Y \in X \Rightarrow Y \cup {Y} \in X)))$

\underline{Example}: $\{ \varnothing, \{\varnothing\}, \{\varnothing, \{\varnothing\}\}, ... \}$

\subsection{Separation}

Given a set ($X$), selecting elements that satisfy a property ($p$) produces a set ($Y$)

$\forall X \exists Y \forall Z (Z \in Y \Leftrightarrow (Z \in X \wedge p(x)))$

\section{Set Operations}

\subsection{Intersection ($\bigcap S$)}

Let $S$ be a set of sets. The intersection of sets in $S$ is the set $T$ that contains elements that belong to all the sets in $S$

$\forall S \exists T \forall Y ((Y \in T) \Leftrightarrow \forall Z ((Z \in S) \Rightarrow (Y \in Z)))$

\begin{itemize*}
	\item $A \cap \varnothing = \varnothing$
	\item $A \cap B = B \cap A$
	\item $A \cap (B \cap C) = (A \cap B) \cap C$
	\item $A \subset B \Leftrightarrow A \cap B = A$
	\item $A \cap (B \cup C) = (A \cap B) \cup (A \cap C)$
	\item $A \cup (B \cap C) = (A \cup B) \cap (A \cup C)$
\end{itemize*}

\subsection{Disjoint}

2 sets are disjoint iff $S \cap T = \varnothing$

\subsubsection{Mutally Disjoint}

Let S be a set of sets. All sets $T \in S$ are disjoint if every 2 different sets are disjoint. "Theres no intersection between sets"

$\forall X \in S \forall Y \in S (X \neq Y \Rightarrow X \cap Y = \varnothing)$

\subsubsection{Partition}

Let $S$ be a set, $V$ be a set of non-empty subsets in $S$. Then $V$ is a partition of $S$ iff

\begin{itemize*}
	\item Sets $V$ are mutually disjoint
	\item Union of sets in V equals S
\end{itemize*}

\subsection{Difference}

\subsubsection{Non Symmetric ($\backslash$)}

$A \backslash B$

"In A, not in B"

\textbf{Complement}

$\overline{T}^S$ is the complement of $T$ in $S$ ($S \backslash T$)

$\forall X (X \in A \backslash B \Leftrightarrow (X \in A \wedge X \notin B))$

\subsubsection{Symetric Difference ($-$)}

"Elements that belong in one of the sets but not both"

$\forall X (X \in (A - B) \Leftrightarrow (X \in A) \oplus X \in T)$

\begin{itemize*}
	\item $\oplus$: XOR
\end{itemize*}


\end{spacing}

\end{document}
