\documentclass[12pt]{article}

\usepackage[T1]{fontenc}
\usepackage{setspace}
\usepackage[a4paper, margin=1.2cm, bottom=2.5cm]{geometry}
\usepackage{graphicx}
\usepackage{mdwlist}
\usepackage{color, soul}

\graphicspath{{./_img/}}
\setlength{\parindent}{0cm}

\title{\textbf{Minimum Spanning Tree (MST)}}
\date{}

\begin{document}

\maketitle

\begin{spacing}{1.5}

\section{MST}

\begin{itemize*}
	\item Spanning Tree of G with min total weight
\end{itemize*}
 
\subsection{Prims ($O(E\log{V})$)}

\subsubsection{Pseudocode}

\begin{verbatim}
visited = new bool[V]
foreach (neighbour in neighbours(src))
    PQ.enqueue(neighbour) // { weight, neighbourIndex }
while (PQ.size() > 0)
    v = PQ.dequeue() // dequeue from PQ (O(log(V))
    if (!visited[v]) // process each edge once (O(E))
        MST.add(v)
        foreach (neighbour in neighbours(v))
            PQ.enqueue(neighbour) // insert into PQ (O(log(V))
\end{verbatim}

\subsection{Kruskal's ($O(E\log{V})$)}

\subsubsection{Pseudocode}

\begin{verbatim}
processed = new bool[E]
foreach (edge in edgeList) // edge list sorted by edge weight (PQ)
    if (!MST.contains(edge) && adding edge does not form cycle)
        MST.add(edge)
\end{verbatim}

\end{spacing}

\end{document}
