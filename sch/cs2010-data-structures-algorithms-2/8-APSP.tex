\documentclass[12pt]{article}

\usepackage[T1]{fontenc}
\usepackage{setspace}
\usepackage[a4paper, margin=1.2cm, bottom=2.5cm]{geometry}
\usepackage{graphicx}
\usepackage{mdwlist}
\usepackage{color, soul}

\graphicspath{{./_img/}}
\setlength{\parindent}{0cm}

\title{\textbf{All Pairs Shortest Path}}
\date{}

\begin{document}

\maketitle

\begin{spacing}{1.5}

\section{Floyd Warshall's ($O(V^3)$)}

\begin{verbatim}
for (intermediate in V)
    for (src in V)
        for (dest in V)
            dist[src][dest] = min(
                dist[src][dest], 
                dist[src][intermediate] + dist[intermediate][dest])
\end{verbatim}

\subsection{Print actual SP}

Using predecessor matrix. Where \texttt{p[i][j]} is last vertex before \texttt{j}

\begin{verbatim}
if (dist[src][intermediate] + dist[intermediate][dest] < dist[src][dest])
    dist[src][dest] = dist[src][intermediate] + dist[intermediate][dest]
    p[src][dest] = intermediate
\end{verbatim}

\subsection{Transitive Closure Problem}

Determine if vertex is connected to another. 

\begin{verbatim}
connected[i][j] = connected[i][j] | connected[i][k] & connected[k][j]
\end{verbatim}

\subsection{Minimax/Maximin}

Minimax: finding minimum of maximum edge weight along all possible paths from one vertex to another. 

\begin{verbatim}
dist[i][j] = min(
        dist[i][j],
        max(dist[i][k], dist[k][j])
    )
\end{verbatim}

\subsection{Detect Any/-ve Cycle}

Set the diagonal to INF, after Floyd Warshall, recheck diagonal. If its -ve, it means theres a -ve cycle. If its not infinity, it means theres a cycle



\end{spacing}

\end{document}
